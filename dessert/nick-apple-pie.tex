\subsubsection{Nick Estrada's apple pie}

\subsubsection{Pie crust}
\begin{align*}
    \ing{1~C}{sifted flour}             & \ing{2~t}{salt} \\
    \ing{\nicefrac{1}{2}~C}{shortening} & \ing{\nicefrac{1}{4}~C}{cold water} \\
\end{align*}

\begin{description}
    \item[Sift]flour into large mixing bowl.
    \item[Cut]in shortening.
    \item[Add]salt.
    \item[Mix]while slowly adding cold water. You will use less than the \nicefrac{1}{4}~C, just keep adding until the consistency is dough, but not soggy.
    \item[Chill]for one~houre before rolling out.
\end{description}

Makes 1 pie crust.

\subsubsection{Apple filling}
\begin{align*}
    \ing{5}{goodly-sized Granny Smith apples} & \ing{\nicefrac{3}{4}~C}{sugar} \\
    \ing{2~T}{flour}                          & \ing{1~T}{cinnamon}            \\
    \ing{2~t}{ginger}                         & \ing{1~t}{nutmeg}              \\
    \ing{1~t}{clove}
\end{align*}

\begin{description}
    \item[Skin]apples and cut them into small pieces.
    \item[Add]the sugar and flour and mix.
    \item[Add]the cinnamon, ginger, nutmeg,\footnote{Like Malcolm X got high on in prison.} and clove.\footnote{The important thing with the cinnamon, ginger, and nutmeg is that they need to coat the apples. More might be needed depending on the size of the apples.}
\end{description}

Bake at \degr{350} for 40~minutes with tin foil covering the edges. Remove foil and cook for an additional 20~minutes.

Eat with ice cream.

\pagebreak
